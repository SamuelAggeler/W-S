\section{Grundlagen}
\subsection[]{Wahrscheinlichkeitsräume}
\Def[1.1Sigma-Algebra ] \newline
Eine Sigma-Algebra ist eine Teilmenge \(\mathcal{F} \subset \mathcal{P}(\Omega)\) mit folgenden Eigenschaften
\begin{itemize}
    \item \(\Omega \in \mathcal{F}\)
    \item  \(A \in \mathcal{F} \implies A^c \in \mathcal{F}\)
    \item \(A_1, A_2, \dots \in \mathcal{F} \implies \bigcup_{i=1}^{\infty} A_i \in \mathcal{F} \)
\end{itemize}
\Def[1.2 Wahrscheinlichkeitsmass ] \newline
Ein Wahrscheinlichkeitsmass ist ein Mapping 
\[ \mathbb{P}: \mathcal{F} \rightarrow [0,1]\]
\[ A \rightarrow \mathbb{P}[A]\]
\begin{itemize}
    \item \( \mathcal{P}[\Omega] = 1\)
    \item \(\sigma-\text{Additivität} \ \mathcal{P}[A] = \sum_{i=1}^{\infty} \mathcal{P}[A_i]\) \newline if \( A = \bigcup_{i=1}^{\infty} A_i \) (disjunkte Vereinigung)
\end{itemize}
\Def[1.3 Wahrscheinlichkeitsraum ] \\
Sei \( \omega \) ein Grundraum, \(\mathcal{F}\) eine \(\sigma\) -Algebra und \(\mathcal{P}\) ein Wahrscheinlichkeitsmass. Wir nennen das Tripel\( (\omega, \mathcal{F}, \mathcal{P})\)  Wahrscheinlichkeitsraum.
\Def[1.5 Laplace Modell] \\
\begin{itemize}
    \item \(\mathcal{F} = \mathcal{P}(\omega)\)
    \item \(\mathbb{P} : \rightarrow \left[0,1\right]\) ist definiert durch \[ \forall A \in \mathcal{F} \ \mathbb{P}[A] = \frac{\abs{A}}{\abs{\omega}}\]
\end{itemize}
\subsection{Eigenschaften von Ereignissen}
\Satz[1.6]
Für eine Sigma-Algebra \(\mathcal{F} \ \text{auf} \ \Omega\) gilt:
\begin{itemize}
    \item \(\emptyset \in \mathcal{F}\)
    \item \(A_1, A_2, \dots \in \mathcal{F} \implies \bigcap_{i=1}^{\infty} A_i \in \mathcal{F}\)
    \item \(A,B \in \mathcal{F} \implies A \cup B \in \mathcal{F}\)
    \item \(A,B \in \mathcal{F} \implies A \cap B \in \mathcal{F}\)
\end{itemize}
\subsection{Eigenschaften von Wahrscheinlichkeitsmassen}
\Satz[1.7]
\begin{itemize}
    \item \( \mathbb{P}[\emptyset] = 0\)
    \item \(A_1, \dots A_k\) paarweise disjunkte Ereignisse, \[\mathbb{P}[A_1 \cup \dots \cup A_k] = \mathbb{P}[A_1] + \dots \mathbb{P}[A_k]\]
    \item \( \mathbb{P}[A^c] = 1 - \mathbb{P}[A]\)
    \item \(\mathbb{P}[A \cup B ] = \mathbb{P}[A] + \mathbb{P}[B] - \mathbb{P}[A \cap B]\)
\end{itemize}
\Bem[1.6A] \newline
\[\left(\bigcup_{i=1}^{\infty}A_i\right)^c = \bigcap_{i=1}^{\infty}(A_i)^c\]
\Satz[1.8] Seien \(A,B \in \mathcal{F}\) dann gilt \[ A \subset B \implies \mathbb{P}[A] \leq \mathbb{P}[B]\]
\Satz[1.9] Sei \( A_1, A_2, \dots \) eine Folge von nicht notwendigerweise disjunkten Ereignissen, dann gilt: \[ \mathbb{P}[\bigcup_{i=1}^{\infty} A_i] \leq \sum_{i=1}^{\infty} \mathbb{P}[A_i]\]
\newline \newline \newline
\subsection{Bedingte Wahrscheinlichkeit}
\Def[1.13 Bedingte Wahrscheinlichkeit] \newline
Sei \((\omega, \mathcal{F}, \mathbb{P})\) ein Wahrscheinlichkeitsraum. Seien A, B zwei Ereignisse mit \( \mathbb{P}[B] > 0\) \[\mathbb{P}[A|B] = \frac{\mathbb{P}\left[A \cap B\right]}{\mathbb{P}\left[B\right]}\]
\Satz[1.16 Gesetz der totalen Wahrscheinlichkeit] \newline
Sei  \( B_1, \dots , B_n\) eine Partition des Grundraumes \( \omega \), so dass \( \mathbb{P}[B_i] > 0 \) für jedes \( 1 \leq i \leq n\) gilt. Dann gilt: \[\forall A \in  \mathcal{F} \ \mathbb{P}[A] = \sum_{i=1}^{n} \mathbb{P} \left[A | B_i \right] \mathbb{P}[B_i]\]
Alternativ : \[P(A) = \mathbb{P}[A \cap B ] + \mathbb{P}[A \cap \overline{B}] = \] 
\[\mathbb{P}[B] \cdot \mathbb{P}[A|B] + \mathbb{P}[\overline{B}] \cdot \mathbb{P}[A|\overline{B}]\]
\Satz[1.17 Satz von Bayes] \newline
Sei \( B_1 \dots B_n \in \mathcal{F }\) eine Partition von \( \omega\) sodass, \( \mathbb{P}[B_i] > 0 \) für jedes i gilt. Für jedes Ereignis A mit \( \mathbb{P}[A] > 0 \) gilt  \[ \forall i = 1, \dots n \ \mathbb{P}\left[ B_i | A \right] = \frac{\mathbb{P}[A | B_i] \mathbb{P}[B_i]}{\sum_{j=1}^{n} \mathbb{P}[A | B_j] \mathbb{P}[B_j] }\]
Alternativ \[\mathbb{P}[A|B] = \frac{\mathbb{P}[B|A] \cdot \mathbb{P}[A]}{\mathbb{P}[B]}\]
\subsection{Unabhängigkeit}
\Def[1.18 Unabhängigkeit] \newline
Sei \( (\omega, \mathcal{F}, \mathbb{P})\) ein Wahrscheinlichkeitsraum. Zwei Ereignisse A und B heissen unabhängig falls \[ \mathbb{P} \left[A \cap B \right] = \mathbb{P}\left[A\right] \mathbb{P} \left[B\right]\]
\Satz[1.20] \newline
Seien A,B \( \in \mathcal{F}\) zwei Ereignisse mit \( \mathbb{P}[A], \mathbb{P}[B] > 0 \). Dann sind folgende Aussagen äquivalent:
\begin{enumerate}
    \item \( \mathbb{P}[A \cap B] = \mathbb{P}[A] \mathbb{P}[B]\)
    \item \( \mathbb{P}[A | B] = \mathbb{P}[A]\)
    \item \( \mathbb{P}[B | A] = \mathbb{P}[B]\)
\end{enumerate}
\Def[1.21]  \newline
Sei I eine beliebige Indexmenge. Eine Familie von Ereignissen \( (A_i)_{i \in I}\) heisst unabhängig falls \[ \forall J \subset I \text{endlich} \quad \mathbb{P}[\bigcap_{j \in J}A_j] = \prod_{j \in J} \mathbb{P}[A_j]\]
\Bem \newline
Drei Ereignisse A,B und C sind unabhängig falls alle 4 folgenden Gleichungen erfüllt sind
\begin{enumerate}
    \item \( \mathbb{P}[A \cap B ] = \mathbb{P}[A] \mathbb{P}[B]\)
    \item \( \mathbb{P}[A \cap C ] = \mathbb{P}[A] \mathbb{P}[C]\)
    \item \( \mathbb{P}[B \cap C ] = \mathbb{P}[B] \mathbb{P}[C]\)
    \item \( \mathbb{P}[A \cap B \cap C  ] = \mathbb{P}[A] \mathbb{P}[B] \mathbb{P}[C]\)
\end{enumerate}