\sep
\Def[2.1] \newline
Sei \( \alpha \in [0,1]\) Ein Konfidenzintervall für \( \theta \) mit Niveau \(1 - \alpha \) ist ein Zufallsintervall I = \([A,B ]\), sodass gilt
\[ \forall \theta \in \Theta \ \mathbb{P}_\theta[A \leq \theta \leq B ] \geq 1 - \alpha \]
wobei A,B Zufallsvariablen der Form \( A = a(X_1, \dots , X_n ), B = b(X_1, \dots , X_n )\) mittels \( a,b : \R^n \rightarrow \R \) sind.
\Bem[2.2] \newline
In der obigen Gleichung ist der Parameter  \( \theta \) deterministisch und nicht zufällig. Die stochastischen Elemente sind gerade die Schranken \( A = a(X_1, \dots , X_n )\) und \( B = b(X_1, \dots , X_n)\)
\Def[2.3] \newline
Eine stetige Zufallsvariable X heisst \( \chi^2-\) Verteilt mit m Freiheitsgraden falls ihre Dichte gegeben ist durch \[ f_X(y) = \frac{1}{2^{\frac{m}{2}}\Gamma(\frac{m}{2})y^{\frac{m}{2}-1e^{-\frac{1}{2}y}}} \text{für y} \geq 0\]
Dabei ist die Gamma-Funktion für \( v \geq 0\) definiert durch \[ \Gamma(v) := \int_0^\infty t^{v-1}e^{-t}dt\]
ES gilt \(\Gamma(n) = (n-1)! \) für \(v = n \in \N \)
\Bem[2.3A] 