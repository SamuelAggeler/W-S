\sep
\Def[4.1] \newline
Sei \( X : \omega \rightarrow \R_+ \) eine Zufallsvariable mit nicht-negativen Werten. Dann heisst \[ \mathbb{E}[X] = \int_0^\infty (1-F_X(x)) dx\] der Erwartungswert von X. \newline
\Bem[4.2] \newline
Der Erwartungswert kann sowohl endliche als auch nicht endliche Werte annehmen. Für allgemeinen ZV. definieren wir den Erwartungswert durch Zerlegen in einen positiven und negativen Teil. Die Beiden Zufallsvariablen sind wie folgt definiert : \[X_+(\omega) = \begin{cases}
    X(\omega) & \text{falls} X(\omega) \geq 0 \\
    0 & \text{falls} X(\omega) < 0
\end{cases}\]
und \[X_-(\omega) = \begin{cases}
    -X(\omega) & \text{falls} X(\omega) \leq 0 \\
    0 & \text{falls} X(\omega) > 0
\end{cases}\]
\(X_+ \) und \(X_-\) sind nicht negative Zufallsvariablen. Zudem gilt \(X = X_+ - X_-\) als auch \( \abs{X} = X_+ + X_-\)
\Satz[4.3] \newline
Sei X eine nicht-negative Zufallsvariable. Dann gilt \[ \mathbb{E}[X] \geq 0\]
\Def[4.4] Sei X eine Zufallsvariable. Falls \(\mathbb{E}[\abs{X}] < \infty \), dann heisst \[ \mathbb{E}[X] = \mathbb{E}[X_+] - \mathbb{E}[X_-]\] Erwartungswert von X. \newline
\Bsp[4.5] \newline Sei X eine Bernoulli Z.V mit Parameter p \[\mathbb{E}[X] = p\] Sei X Poisson-verteilt mit Parameter \( \lambda > 0\) \[\mathbb{E}[X] = \lambda\] Sei \(A \in \mathcal{F}\) ein Ereignis. Sei \( \mathbbm{1}_A \) die Indikatorfunktion auf A, \[ \forall \omega \in \Omega \ \mathbbm{1}_A(\omega) = \begin{cases}
    0 & \text{falls} \omega \notin A \\
    1 & \text{falls} \omega \in A
\end{cases}\]
Dann ist \(\mathbbm{1}_A\) eine Zufallsvariable. Per Definition gilt \[\mathbb{P}[X = 0] = 1 - \mathbb{P}[A] \ \text{und} \ \mathbb{P}[X=1] = \mathbb{P}[A]\]
Somit ist \(\mathbbm{1}_A\) eine Bernoulli Z.V mit Paramter \( \mathbb{P}[A] \ \implies \mathbb{E}[\mathbbm{1}_A] = \mathbb{P}[A]\) \newline
\Satz[4.6] Sei \( X: \omega \rightarrow \R \) eine diskrete Zufallsvariable dessen Werte in W (endlich oder abzähbar ) fast sicher liegen.Dann gilt \[ \mathbb{E}[X] = \sum_{x \in W} x \cdot \mathbb{P}[X = x]\]
\Satz[4.7] Sei \( X: \omega \rightarrow \R \) eine diskrete Zufallsvariable mit Werten in W (endlich oder abzähbar). Für jedes \(\phi : \R \rightarrow \R \) gilt \[ \mathbb{E}[\phi(X)] = \sum_{x \in W} \phi(x) \mathbb{P}[(X = x)]\]
\Satz[4.8] Sei X eine stetige Zufallsvariable mit Dichte f. Dann gilt \[\mathbb{E}[X] = \int_{-\infty}^{\infty} x \cdot f(x)dx\]
\Bsp[4.8A] Gleichverteilung auf \([a,b]\) \[\mathbb{E}[X] = \frac{a + b}{2}\] Exponentialverteilung mit \( \lambda > 0 \) \[\mathbb{E}[X] = \frac{1}{\lambda}\]
\Theo[4.9] Sei X eine stetige Zufallsvariable mit Dichte f. Sei \(\phi : \R \rightarrow \R \) eine Abbildung, sodass \( \phi(X )\) eine Zufallsvariable ist. Dann gilt \[ \mathbb{E}[\phi(X)] = \int_{-\infty}^{\infty} \phi(x)f(x)dx\]
\Theo[4.10 linearität des Erwartungswert] \newline
Seien \(X, Y : \omega \rightarrow \R \) Zufallsvariablen, sei \(\lambda \in \R\). Falls die Erwartungswerte wohldefiniert sind gilt \begin{itemize}
    \item \(\mathbb{E}[ \lambda \cdot X ] = \lambda \cdot \mathbb{E}[X]\)
    \item \(\mathbb{E}[X + Y] = \mathbb{E}[X] + \mathbb{E}[Y]\)
\end{itemize}
\Bem[4.11] \newline 
Die Zufallsvariablen müssen dabei nicht unabhängig sein. \newline
\Bem[4.12] \newline
Anwendung1 : Der Erwartungswert einer binomialverteilten Zufallsvariable. Sei \( n \geq 1\) und \( 0 \leq p \leq 1 \). Sei S eine binomialverteilte Zufallsvariable mit Parametern n und p. \[\mathbb{E}[S] = \sum_{k=0}^{n} \cdot \binom{n}{k} p^k (1-p)^{n-k}\]
Wir können allerdings davon ausgehen, dass S die gleiche Verteilung hat wie \(S_n = X_1 + \dots + X_n\) wobei \(X_1 , \dots , X_n \) u.i.v Bernoulli Z.V mit Parameter p sind. Durch anwendung der linearität erhalten wir \[ \mathbb{E}[S] = \mathbb{E}[S_n] = np\]
Anwendung2 : Rechnen mit Normalverteilungen mit Parameter m und \( \sigma^2\). Wenn X eine Normalverteilung mit Parametern m und \( \sigma^2\) ist, dann hat sie die gleiche Verteilung wie \( m + \sigma \cdot Y\), wobei Y eine standardnormalverteilte Z.V ist. Aus der Linearität des Erwartungswertes folgt \[ \mathbb{E}[X] = \mathbb{E}[m + \sigma \cdot Y] = m + \sigma \mathbb{E}[Y]\] Also müssen wir nur \( \mathbb{E}[Y]\) berechnen. \[\mathbb{E}[Y] = \int_{-\infty}^{\infty} x \cdot f_{0,1}(x) dx = 0\] da \(X x \cdot f_{0,1}(x)\) eine ungerade funktion ist. \[\mathbb{E}[X] = m\]
\Theo[4.13] Seien X,Y zwei Zufallsvariablen. Falls X und Y unabhängig sind, dann gilt \[ \mathbb{E}[XY] = \mathbb{E}[X]\mathbb{E}[Y]\]
\Satz[4.14 Stetige Extremwertformel] \newline
Sei X eine Zufallsvariable, sodass \( X \geq 0 \) fast sicher gilt. Dann gilt \[ \mathbb{E}[X] = \int_0^{\infty} \mathbb{P}[X \geq x]dx\]
\Bem[4.14A]
Anwedungen: Ausrechnen von Erwartungswert einer exponential-verteilten Zufallsvariable. Sei T eine exponential-verteilte Zufallsvariable mit Paramter \(\lambda \geq 0\) \[ \mathbb{E}[T] = \int_0^\infty \mathbb{P}[X \geq x]dx = \int_0^\infty \exp^{-\lambda x }dx = \frac{1}{\lambda}\]
\Satz[4.15 Diskrete Extremwertformel] \newline
Sei X eine diskrete Zufallsvariable mit Werten in \( \N = \{0, 1, 2, \dots \}\). Dann gilt folgende Identität \[ \mathbb{E}[X] = \sum_{n=1}^{\infty} \mathbb{P}[X \geq n]\]
\Satz[4.16] \newline
Sei X eine Zufallsvariable. Sei \( f: \R \rightarrow \R_+ \) eine Abbildung, sodass \( \int_{-\infty}^{+ \infty} f(x)dx = 1\). Dann sind folgende Aussagen äquivalent
\begin{itemize}
    \item X ist stetig mit Dichte f,
    \item Für jede stückweise stetige, beschränkte Abbildung \(\phi: \R \rightarrow \R \) gilt \[ \mathbb{E}[\phi(X)] = \int_{-\infty}^{\infty}\phi (x) f(x)dx\]
\end{itemize}
\Theo[4.17] \newline
Seien X,Y zwei diskrete Zufallsvariablen. Die folgenden Aussagen sind äquivalent
\begin{itemize}
    \item X, Y sind unabhängig
    \item Für jedes \( \phi : \R \rightarrow \R, \psi : \R \rightarrow \R \) beschränkt und stückweise stetig gilt \[ \mathbb{E}[\phi(X)\psi(Y)] = \mathbb{E}[\phi(X)]\mathbb{E}[\psi(Y)]\]
\end{itemize}
\Theo[4.18] \newline
Seien \(X_1, \dots , X_n \) n Zufallsvariablen. Die folgenden Aussagen sind äquivalent
\begin{itemize}
    \item \(X_1, \dots , X_n \) sind unabhängig
    \item Für jedes \( \phi_1 (X_1), \dots \phi_n(X_n ) : \R \rightarrow \R \) beschränkt gilt \[\mathbb{E}[\phi_1(X_1), \dots \phi_n(X_n)] = \mathbb{E}[\phi_1(X_1)] \dots \mathbb{E}[\phi_n(X_n)]\]
\end{itemize}
\Satz[4.19] \newline
Seien X,Y zwei Zufallsvariablen, sodass \[ X \leq Y f.s\] gilt. Falls beide Erwartungswerte wohldefiniert sind folgt dann \[ \mathbb{E}[X] \leq \mathbb{E}[Y] f.s\]
\Theo[4.20 Markow-Ungleichung] \newline
Sei X eine nicht-negative Zufallsvariable. Für jedes \( a > 0\) gilt dann \[ \mathbb{P}[X \geq a ] \leq \frac{\mathbb{E}[X]}{a}\]
\Theo[4.21 Jensen Ungleichung] \newline
Sei X eine Zufallsvariable. Sei \( \phi \R \rightarrow \R \) eine konvexe Funktion. Falls \( \mathbb{E}[\phi(X)]\) und \( \mathbb{E}[X]\) wohldefiniert sind, gilt \[ \phi(\mathbb{E}[X] \leq \mathbb{E}[\phi(X)])\]
\Def[4.22] Sei X eine Zufallsvariable, sodass \( \mathbb{E}[X^2] < \infty \). Wir definieren die Varianz von X durch \[ \sigma_X^2 = \mathbb{E}[(X-m)^2], \ \text{wobei} \ m=\mathbb{E}[X]\] DIe Wurzel aus \(\sigma_X^2\) nennen wir die Standardabweichung von X \newline
\Satz[4.24 Chebychev Ungleichung] \newline
Sei X eine Zufallsvariable mit \( \mathbb{E}[X^2] < \infty\). Dann gilt für jedes \( a \geq 0\) \[ \mathbb{P}[\abs{X -m} \geq a] \leq \frac{\sigma_X^2}{a^2}, \text{wobei m} = \mathbb{E}[X]\]
\Satz[4.25] Grundlegende Eigenschaften der Varianz \newline
\begin{itemize}
    \item Sei X eine Zufallsvariable mit \( \mathbb{E}[X^2] < \infty \). Dann gilt \[ \sigma_X^2 = \mathbb{E}[X^2] - \mathbb{E}[X]^2\]
    \item Sei X eine Zufallsvariable mit \( \mathbb{E}[X^2] < \infty \) und sei \( \lambda \in \R \). Dann gilt \[ \sigma_{\lambda X}^2 = \lambda^2 \cdot \sigma_X^2\]
    \item Seien \(X_1, \dots , X_n \) n-viele paarweise unabhängige Zufallsvariablen und \(S = X_1 + \dots + X_n\). Dann gilt \[ \sigma_S^2 = \sigma_{X_1}^2 + \dots + \sigma_{X_n}^2\]
\end{itemize}
\Def[4.26] \newline
Seien X,Y zwei Zufallsvariablen mit endlichen zweiten Momenten \( \mathbb{E}[X^2] < \infty \) und \( \mathbb{E}[Y^2] < \infty \). Wir definieren die Kovarianz zwischen X und Y durch \[\text{Cov}(X,Y) = \mathbb{E}[XY]-\mathbb{E}[X]\mathbb{E}[Y]\]
\Bem[4.26A] \newline
Es gilt X,Y unabhängig \( \implies \text{Cov}(X,Y) = 0\)
